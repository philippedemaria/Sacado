%% Théorème, Définition, Exemple, Remarque, .....
\newcounter{cpttheo}
\setcounter{cpttheo}{0}
\newcounter{cptdef}
\setcounter{cptdef}{0}
\newcounter{cptmth}
\setcounter{cptmth}{0}
\newcounter{cpttitre}
\setcounter{cpttitre}{0}
\newcounter{cpttitrebis}
\setcounter{cpttitrebis}{0}
 % Exercices
\newcounter{cptapp}
\setcounter{cptapp}{0}
\newcounter{cptex}
\setcounter{cptex}{0}
\newcounter{cptsr}
\setcounter{cptsr}{0}
\newcounter{cpti}
\setcounter{cpti}{0}
\newcounter{cptcor}
\setcounter{cptcor}{0}
\newcounter{cptcahier}
\setcounter{cptcahier}{0}
\newcounter{cptavdm}
\setcounter{cptavdm}{0}
\newcounter{cptpara}
\setcounter{cptpara}{0}
\newcounter{subcptpara}
%%%%% Pour réinitialiser numéros des paragraphes après une nouvelle partie
\makeatletter
    \@addtoreset{paragraph}{part}
\makeatother


%%%% Titres et sections

\newlength\chapnumb
\setlength\chapnumb{3cm}


%\titleformat{\part}[block] {
%  \normalfont\sffamily\color{violet}}{}{0pt} {
%    \parbox[t]{\chapnumb}{\fontsize{120}{110}\selectfont\ding{110}}
%    \parbox[b]{\dimexpr\textwidth-\chapnumb\relax}{
%        \raggedleft
%        \hfill{{\color{bleu3}\fontsize{40}{30}\selectfont#1}}\\
%        \rule{0.99\textwidth-\chapnumb\relax}{0.4pt}
%  }
%}
 
\titleformat{name=\part,numberless}[block]
{\normalfont\sffamily\color{bleu3}}{}{0pt}
{\parbox[b]{\chapnumb}{%
   \mbox{}}%
  \parbox[b]{\dimexpr\textwidth-\chapnumb\relax}{%
    \raggedleft%
    \hfill{{\color{bleu3}\fontsize{40}{30}\selectfont#1}}\\
    \rule{0.99\textwidth-\chapnumb\relax}{0.4pt}
  }
}


 
\titleformat{\chapter}[block] {
  \normalfont\sffamily\color{violet}}{}{0pt} {
    \parbox[t]{\chapnumb}{ 
      \fontsize{120}{110}\selectfont\thechapter}
     \parbox[b]{\dimexpr\textwidth-\chapnumb\relax}{
        \raggedleft
        \hfill{{\color{bleu3}\huge#1}}\\  
  \ifthenelse{\thechapter<10}{\rule{0.99\textwidth-\chapnumb\relax}{0.4pt}}{\rule{0.9\textwidth-\chapnumb\relax}{0.4pt}}
        \setcounter{cpttitre}{0}
        \setcounter{cptpara}{0}
  }
}
% 
%\titleformat{name=\chapter,numberless}[block]
%{\normalfont\sffamily\color{bleu3}}{}{0pt}
%{\parbox[b]{\chapnumb}{%
%   \mbox{}}%
%  \parbox[b]{\dimexpr\textwidth-\chapnumb\relax}{%
%    \raggedleft%
%    \hfill{{\color{bleu3}\huge#1}}\\
%    \ifthenelse{\thechapter<10}{\rule{0.99\textwidth-\chapnumb\relax}{0.4pt}}{\rule{0.9\textwidth-\chapnumb\relax}{0.4pt}}
%    \setcounter{cpttitre}{0}
%  }
%}

% couleurs :   bleu3 - eduscol4P - black - eduscol4B   

  \titleformat{\section}[hang]{\color{black} \normalfont\filright\Large}{}{0.4em}{#1}   
  \titlespacing*{\section}{0.2pt}{0ex plus 0ex minus 0ex}{0.3em}
    
  \titleformat{\subsection}[hang]{\color{violet} \normalfont\filright\Large}{}{0 em}{#1}            
  \titleformat{\subsubsection}[hang]{\color{violet}  \normalfont\filright\normalsize}{ }{0 em}{#1}
  \titleformat{\paragraph}[hang]{\color{violet} \normalfont\filright\normalsize}{}{0 em}{#1}


\newcommand{\paragraphe}[1]{\vspace{0.4cm} \color{eduscol4B}{\normalfont\filright\large{ \stepcounter{cptpara}  \thecptpara.  #1}}  \color{black} \vspace{0.2cm} 
\setcounter{subcptpara}{1} } 

%eduscol4P

\newcommand{\subparagraphe}[1]{\vspace{0.4cm} \color{eduscol4B}{\normalfont\filright\normalsize{ %
  
  \thecptpara. \thesubcptpara.    #1 }}  \color{black} \vspace{0.2cm} 
  \stepcounter{subcptpara}
  } 


%%%%%%%%%%%%%%%%%%%%%% Cycle 4
\newcommand{\TitreSansTemps}[1]{\section*{#1} 
\setcounter{cptpara}{0}
}



\newcommand{\Titre}[2]{\section*{#1 
\ifthenelse{\equal{#2}{1}}   {\hfill{  \begin{small}
   1 heure 
   \end{small} } \addcontentsline{toc}{section}{#1} }%
{% sinon
\ifthenelse{\equal{#2}{0}}   {\hfill{  } \addcontentsline{toc}{section}{#1} }%
{ 
 {\hfill{ \begin{small}
   #2 heures 
   \end{small}} \addcontentsline{toc}{section}{#1} }%
}%
}%
}%
\setcounter{cptpara}{0} 
}

\newcommand{\Rituel}[2]{\section*{#1 
\ifthenelse{\equal{#2}{1}}   {\hfill{  \begin{small}
   #2 minute 
   \end{small} } \addcontentsline{toc}{section}{#1} }%
{% sinon
\ifthenelse{\equal{#2}{0}}   {\hfill{  } \addcontentsline{toc}{section}{#1} }%
{ 
 {\hfill{ \begin{small}
   #2 minutes 
   \end{small}} \addcontentsline{toc}{section}{#1} }%
}%
}%
}%
\setcounter{cptpara}{0} 
}

%%%%%%%%%%%%%%%%%%%%%% Cycle 4
\newcommand{\TitreDate}[2]{\section*{#1 
\ifthenelse{\equal{#2}{1}}   {\hfill{  \begin{small}
   1 heure 
   \end{small} } \addcontentsline{toc}{section}{#1} }%
{% sinon
\ifthenelse{\equal{#2}{0}}   {\hfill{  } \addcontentsline{toc}{section}{#1} }%
{ 
 {\hfill{ \begin{small}
   #2 
   \end{small}} \addcontentsline{toc}{section}{#1} }%
}%
}%
}%
\setcounter{cpttitre}{1}
\setcounter{cpttitrebis}{1}
\setcounter{cptpara}{0}
\setcounter{cptavdm}{1}
}

%%%%%%%%%%%%% Fiche AP
\newenvironment{synthese}[3][]{%
\vspace{0.5cm}
\begin{tcolorbox}[enhanced, lifted shadow={0mm}{0mm}{0mm}{0mm}%
{black!60!white}, attach boxed title to top left={xshift=110mm, yshift*=-3mm}, coltitle=violet, colback=violet!25!white, boxed title style={colback=white!100}, colframe=violet,title= \textbf{Synthèse #1 #2}. #3  ]}
{%
\end{tcolorbox}
\par}
 

%%%%%%%%%%%%% Séance
\newenvironment{synthesecours}[2][]{%
\newpage
\begin{tcolorbox}[enhanced, lifted shadow={0mm}{0mm}{0mm}{0mm}%
{black!60!white}, attach boxed title to top left={xshift=90mm, yshift*=-3mm}, sharpish corners, coltitle=eduscol4P, colback=eduscol4P!25!white, boxed title style={colback=white!100}, colframe=eduscol4P,title= Synthèse. \quad #1 #2 ]}
{%
\end{tcolorbox}
\par}
%%%%%%%%%%%%% Séance
\newenvironment{seance}[2][]{%
\newpage
\setcounter{cptex}{0}
\setcounter{cptavdm}{0}
\stepcounter{cpttitre}
\begin{tcolorbox}[enhanced, lifted shadow={0mm}{0mm}{0mm}{0mm}%
{black!60!white}, attach boxed title to top left={xshift=90mm, yshift*=-3mm}, sharpish corners, coltitle=eduscol4P, colback=eduscol4P!25!white, boxed title style={colback=white!100}, colframe=eduscol4P,title= Séance \thecpttitre. \quad #1 #2 ]}
{%
\end{tcolorbox}
\par}

%%%%%%%%%%%%% Fiche enseignant Séance
\newenvironment{seanceprof}[2][]{%
\newpage
\setcounter{cptex}{0}
\stepcounter{cpttitrebis}
\begin{tcolorbox}[enhanced, lifted shadow={0mm}{0mm}{0mm}{0mm}%
{black!60!white}, attach boxed title to top left={xshift=70mm, yshift*=-3mm}, sharpish corners, coltitle=eduscol4B, colback=eduscol4B!25!white, boxed title style={colback=white!100}, colframe=eduscol4B,title= Fiche enseignant - Séance \thecpttitrebis. \quad #1 #2 ]}
{%
\end{tcolorbox}
\par}

%%%%%%%%%%%%% Fiche enseignant Séance
\newenvironment{seanceTice}[2][]{%
\newpage
\setcounter{cptex}{0}
\setcounter{cptavdm}{0}
\stepcounter{cpttitrebis}
\begin{tcolorbox}[enhanced, lifted shadow={0mm}{0mm}{0mm}{0mm}%
{black!60!white}, attach boxed title to top left={xshift=70mm, yshift*=-3mm}, sharpish corners, coltitle=gray, colback=gray!25!white, boxed title style={colback=white!100}, colframe=gray,title= TICE \thecpttitrebis. \quad #1 #2 ]}
{%
\end{tcolorbox}
\par}
%%%%%%%%%%%%% Fiche AP
\newenvironment{ficheAP}[3][]{%
\vspace{0.5cm}
\begin{tcolorbox}[enhanced, lifted shadow={0mm}{0mm}{0mm}{0mm}%
{black!60!white}, attach boxed title to top left={xshift=70mm, yshift*=-3mm}, coltitle=violet, colback=violet!25!white, boxed title style={colback=white!100}, colframe=violet,title= \textbf{AP - Fiche #1 #2}. #3  ]}
{%
\end{tcolorbox}
\par}
%%%%%%%%%%%%% Fiche Méthode
\newenvironment{ficheMethode}[2][]{%
\newpage
\begin{tcolorbox}[enhanced, lifted shadow={0mm}{0mm}{0mm}{0mm}%
{black!60!white}, attach boxed title to top left={xshift=90mm, yshift*=-3mm}, sharpish corners, coltitle=eduscol4P, colback=eduscol4P!25!white, boxed title style={colback=white!100}, colframe=eduscol4P,title= Fiche Méthode. \quad #1 #2 ]}
{%
\end{tcolorbox}
\par}
%%%%%%%%%%%%% Séance
\newenvironment{titreDTL}[2][]{%
\newpage
\setcounter{cptex}{0}
\setcounter{cptavdm}{0}
\stepcounter{cpttitre}
\begin{tcolorbox}[enhanced, lifted shadow={0mm}{0mm}{0mm}{0mm}%
{black!60!white}, attach boxed title to top left={xshift=90mm, yshift*=-3mm}, sharpish corners, coltitle=eduscol4P, colback=olive!25!white, boxed title style={colback=white!100}, colframe=olive,title= En temps libre. \quad #1 #2 ]}
{%
\end{tcolorbox}
\par}

%%%%%%%%%%%%% Titre avec numérotation individuelle
\newenvironment{titreInd}[3][]{%
\setcounter{cptex}{0}
\vspace{0.5cm}
\begin{tcolorbox}[enhanced, lifted shadow={0mm}{0mm}{0mm}{0mm}%
{black!60!white}, attach boxed title to top left={xshift=110mm, yshift*=-3mm}, coltitle=violet, colback=violet!25!white, boxed title style={colback=white!100}, colframe=violet,title= \textbf{Fiche #1 #2}. #3  ]}
{%
\end{tcolorbox}
\par}

%%%%%%%%%%%%% Titre DS
\newenvironment{titreDS}[2][]{%
\setcounter{cptex}{0}
\setcounter{cptavdm}{0}
\vspace{0.5cm}
\begin{tcolorbox}[enhanced, lifted shadow={0mm}{0mm}{0mm}{0mm}%
{black!60!white}, attach boxed title to top left={xshift=90mm, yshift*=-3mm}, coltitle=bleu3, colback=bleu1!25!white, boxed title style={colback=white!100}, colframe=bleu3,title=Devoir surveillé #1 #2 ]}
{%
\end{tcolorbox}
\par}

%%%%%%%%%%%%% Titre
\newenvironment{titre}[2][]{%
\setcounter{cptex}{0}
\setcounter{cptavdm}{0}
\vspace{0.5cm}
\begin{tcolorbox}[enhanced, lifted shadow={0mm}{0mm}{0mm}{0mm}%
{black!60!white}, attach boxed title to top left={xshift=90mm, yshift*=-3mm}, coltitle=bleu3, colback=bleu1!25!white, boxed title style={colback=white!100}, colframe=bleu3,title=\stepcounter{cpttitre}   #1 #2 ]}
{%
\end{tcolorbox}
\par}


%%%%%%%%%%%%% Titre
\newenvironment{rituel}[2][]{%
\vspace{0.5cm}
\begin{tcolorbox}[enhanced, lifted shadow={0mm}{0mm}{0mm}{0mm}%
{black!60!white}, attach boxed title to top left={xshift=90mm, yshift*=-3mm}, coltitle=bleu3, colback=white!25!white, boxed title style={colback=white!100}, colframe=bleu3,title= \textbf{Rituel} #1. #2 ]}
{%
\end{tcolorbox}
\par}

%%%%%%%%%%%%% Titre TICE
\newenvironment{titreTice}[2][]{%
\setcounter{cptex}{0}
\setcounter{cptavdm}{0}
\vspace{0.5cm}
\begin{tcolorbox}[enhanced, lifted shadow={0mm}{0mm}{0mm}{0mm}%
{black!60!white}, attach boxed title to top left={xshift=90mm, yshift*=-3mm}, coltitle=bleu3, colback=gray!25!white, boxed title style={colback=white!100}, colframe=gray,title=\stepcounter{cpttitre} \textbf{Fiche \thecpttitre}. #1 #2 ]}
{%
\end{tcolorbox}
\par}

%%%%%%%%%%%%% Thème
\newenvironment{titreTheme}[2][]{%
\setcounter{cptex}{0}
\setcounter{cptavdm}{0}
\vspace{0.5cm}
\begin{tcolorbox}[enhanced, lifted shadow={0mm}{0mm}{0mm}{0mm}%
{black!60!white}, attach boxed title to top left={xshift=90mm, yshift*=-3mm}, coltitle=bleu3, colback=bleu1!25!white, boxed title style={colback=white!100}, colframe=bleu3,title=\stepcounter{cpttitre} \textbf{Thème \thecpttitre}. #1 #2 ]}
{%
\end{tcolorbox}
\par}

%%%%%%%%%%%%% Définitions
\newenvironment{Def}[1][]{%
\setcounter{cptex}{0}
\medskip \begin{tcolorbox}[widget,colback=violet!15,colframe=violet!75!white,
adjusted title= \stepcounter{cptdef} Définition \thecptdef . {#1} ]}
{%
\end{tcolorbox}\par}


\newenvironment{DefT}[2][]{%
\medskip \begin{tcolorbox}[widget,colback=violet!15,colframe=violet!75!white,
adjusted title= \stepcounter{cptdef} Définition \thecptdef . {#1} \textit{#2}]}
{%
\end{tcolorbox}\par}


\newenvironment{DefN}[1][]{%
\setcounter{cptex}{0}
\medskip \begin{tcolorbox}[widget,colback=violet!15,colframe=violet!75!white,
adjusted title= \stepcounter{cptdef} Définition. {#1} ]}
{%
\end{tcolorbox}\par}


%%%%%%%%%%%%% Proposition
\newenvironment{Prop}[1][]{%
\medskip \begin{tcolorbox}[widget,colback=bleu1!15,colframe=bleu1!75!black,
adjusted title= \stepcounter{cptdef} Proposition \thecpttheo . {#1} ]}
{%
\end{tcolorbox}\par}

%%%%%%%%%%%%% Propriétés
\newenvironment{Pp}[1][]{%
\medskip \begin{tcolorbox}[widget,colback=bleu1!15,colframe=bleu1!75!black,
adjusted title= \stepcounter{cpttheo} Propriété \thecpttheo . {#1}]}
{%
\end{tcolorbox}\par}

\newenvironment{PpT}[2][]{%
\medskip \begin{tcolorbox}[widget,colback=bleu1!15,colframe=bleu1!75!black,
adjusted title= \stepcounter{cpttheo} Propriété \thecpttheo . {#1} #2]}
{%
\end{tcolorbox}\par}

\newenvironment{Pps}[1][]{%
\medskip \begin{tcolorbox}[widget,colback=bleu1!15,colframe=bleu1!75!black,
adjusted title= \stepcounter{cpttheo} Propriétés \thecpttheo . {#1}]}
{%
\end{tcolorbox}\par}

%%%%%%%%%%%%% Théorèmes
\newenvironment{ThT}[2][]{% théorème avec titre
\medskip \begin{tcolorbox}[widget,colback=bleu1!15,colframe=bleu1!75!black,
adjusted title= \stepcounter{cpttheo} Théorème \thecpttheo . {#1} #2]}
{%
\end{tcolorbox}\par}

\newenvironment{Th}[1][]{%
\medskip \begin{tcolorbox}[widget,colback=bleu1!15,colframe=bleu1!75!black,
adjusted title= \stepcounter{cpttheo} Théorème \thecpttheo . {#1}]}
{%
\end{tcolorbox}\par}


\newenvironment{ThN}[1][]{%
\medskip \begin{tcolorbox}[widget,colback=bleu1!15,colframe=bleu1!75!black,
adjusted title= \stepcounter{cpttheo} Théorème. {#1}]}
{%
\end{tcolorbox}\par}
%%%%%%%%%%%%% Règles
\newenvironment{Reg}[1][]{%
\medskip \begin{tcolorbox}[widget,colback=bleu1!15,colframe=bleu1!75!black,
adjusted title= \stepcounter{cpttheo} Règle \thecpttheo . {#1}]}
{%
\end{tcolorbox}\par}

%%%%%%%%%%%%% REMARQUES
\newenvironment{Rq}[1][]{%
\begin{bclogo}[couleur=info!30, arrondi =0.15, noborder=true, couleurBarre=info, logo = \bcinfo ]{ 
{\color{info}\normalsize{Remarque#1}}}}
{%
\end{bclogo}
\par}

\newenvironment{Rqs}[1][]{%
\begin{bclogo}[couleur=info!30, arrondi =0.15, noborder=true, couleurBarre=info, logo = \bcinfo ]{ 
{\color{info}\normalsize{Remarques#1}}}}
{%
\end{bclogo}
\par}

%%%%%%%%%%%%% Démonstration
\newenvironment{Dem}[1][]{%
\begin{bclogo}[couleur=white!30, arrondi =0.15, noborder=true, couleurBarre=bleu1, logo = \bccrayon ]{ 
\normalsize{Démonstration.#1}}}
{%
\end{bclogo}
\par}


%%%%%%%%%%%%% Démonstration
\newenvironment{Approfondissement}[1][]{%
\begin{bclogo}[couleur=white!30, arrondi =0.15, noborder=true, couleurBarre=red, logo = \bcstop ]{ 
{\color{red}\normalsize{Approfondissement #1}}}}
{%
\end{bclogo}
\par}

%%%%%%%%%%%%% EXEMPLES
\newenvironment{Ex}[1][]{%
\begin{bclogo}[couleur=white!30, arrondi =0.15, noborder=true, couleurBarre=yellow, logo = \bclampe ]{ 
\normalsize{Exemple#1}}}
{%
\end{bclogo}
\par}

%%%%%%%%%%%%% Rappel
\newenvironment{Rap}[1][]{%
\begin{bclogo}[couleur=bleu1!30, arrondi =0.15, noborder=true, couleurBarre=bleu3, logo = \bcinfo ]{ 
\normalsize{Rappel.#1}}}
{%
\end{bclogo}
\par}

%%%%%%%%%%%%% COUPS DE POUCE
\newenvironment{CdP}[1][]{%
\begin{bclogo}[couleur=bleu1!30, arrondi =0.05, noborder=true, couleurBarre=info, logo = \bclampe ]{ 
{\color{info}\normalsize{Coup de pouce#1}}}}
{%
\end{bclogo}
\par}

%%%%%%%%%%%%% CONTRE EXEMPLES
\newenvironment{CE}[1][]{%
\begin{bclogo}[couleur=red!30, arrondi =0.15, noborder=true, couleurBarre=red, logo = \bcattention ]{ 
\normalsize{Contre exemple#1}}}
{%
\end{bclogo}
\par}

%%%%%%%%%%%%% Preuve
\newenvironment{Pv}[1][]{%
\begin{tcolorbox}[breakable, enhanced,widget, colback=bleu1!10!white,boxrule=0pt,frame hidden,
borderline west={1mm}{0mm}{bleu1!75}]
\textbf{Preuve#1 : }}
{%
\end{tcolorbox}
\par}


%%%%%%%%%%%%% PreuveROC
\newenvironment{PvR}[1][]{%
\begin{tcolorbox}[breakable, enhanced,widget, colback=bleu1!10!white,boxrule=0pt,frame hidden,
borderline west={1mm}{0mm}{bleu1!75}]
\textbf{Preuve (ROC)#1 : }}
{%
\end{tcolorbox}
\par}


%%%%%%%%%%%%% Compétences
\newenvironment{Cps}[1][]{%
\vspace{0.5cm}
\begin{tcolorbox}[enhanced, lifted shadow={0mm}{0mm}{0mm}{0mm}%
{black!60!white}, attach boxed title to top left={xshift=5mm, yshift*=-3mm}, coltitle=white, colback=sapgreen!5!white, boxed title style={colback=sapgreen!100}, colframe=sapgreen!75!black,title=\textbf{Compétences associées#1}]}
{%
\end{tcolorbox}
\par}

%%%%%%%%%%%%% Compétences Collège
\newenvironment{CpsCol}[1][]{%
\vspace{0.5cm}
\begin{tcolorbox}[breakable, enhanced,widget, colback=bleu1!25!white ,boxrule=0pt,frame hidden,
borderline west={2mm}{0mm}{bleu3}]
\textbf{#1}}
{%
\end{tcolorbox}
\par}

%%%%%%%%%%%%% Compétences Collège
\newenvironment{CC}[1][]{%
\vspace{0.5cm}
\begin{tcolorbox}[breakable, enhanced,widget, colback=purple!25!white ,boxrule=0pt,frame hidden,
borderline west={2mm}{0mm}{purple}]
\textbf{#1}}
{%
\end{tcolorbox}
\par}


%%%%%%%%%%%%% paragraphe en sur épaisseur
\newtcolorbox{Ptex}[1][]{enhanced,
  before skip=2mm,after skip=3mm,
  boxrule=0.4pt,left=5mm,right=2mm,top=1mm,bottom=1mm,
  colback=yellow!50,
  colframe=yellow!50!black,
  sharp corners,rounded corners=southeast,arc is angular,arc=3mm,
  underlay={%
    \path[fill=tcbcol@back!80!black] ([yshift=3mm]interior.south east)--++(-0.4,-0.1)--++(0.1,-0.2);
    \path[draw=tcbcol@frame,shorten <=-0.05mm,shorten >=-0.05mm] ([yshift=3mm]interior.south east)--++(-0.4,-0.1)--++(0.1,-0.2);
    \path[fill=yellow!50!black,draw=none] (interior.south west) rectangle node[white]{\Huge\bfseries !} ([xshift=4mm]interior.north west);
    },
  drop fuzzy shadow,#1}



\newtcolorbox{Ptint}[1][]{enhanced,
  before skip=2mm,after skip=3mm,
  boxrule=0.4pt,left=5mm,right=2mm,top=1mm,bottom=1mm,
  colback=green!10,
  colframe=blue!45,
  sharp corners,rounded corners=southeast,arc is angular,arc=3mm,
  underlay={%
    \path[fill=tcbcol@back!50!black] ([yshift=3mm]interior.south east)--++(-0.4,-0.1)--++(0.1,-0.2);
    \path[draw=tcbcol@frame,shorten <=-0.05mm,shorten >=-0.05mm] ([yshift=3mm]interior.south east)--++(-0.4,-0.1)--++(0.1,-0.2);
    \path[fill=blue!40,draw=none] (interior.south west) rectangle node[white]{\Huge\bfseries ?} ([xshift=4mm]interior.north west);
    },
  drop fuzzy shadow,#1}

\newtcbox{\maboite}[1][red]{on line, arc=7pt,colback=#1!10!white,colframe=#1!50!black, before upper={\rule[-3pt]{0pt}{15pt}},boxrule=1pt, boxsep=0pt,left=12pt,right=12pt,top=3pt,bottom=3pt}

%%%%%%%%%%%%% Attendus
\newenvironment{Ats}[1][]{%

\begin{tcolorbox}[enhanced, lifted shadow={0mm}{0mm}{0mm}{0mm}%
{black!60!white}, attach boxed title to top left={xshift=5mm, yshift*=-3mm}, coltitle=white, colback=sapgreen!5!white, boxed title style={colback=sapgreen!100}, colframe=sapgreen!75!black,title=\textbf{Attendus du chapitre#1}]}
{%
\end{tcolorbox}
\par}

%%%%%%%%%%%%% Méthode
\newenvironment{Mt}[1][]{%

\begin{bclogo}[couleur=bleu1!15, arrondi =0.15, noborder=true, couleurBarre=bleu3, logo = \bccrayon ]{ 
\normalsize{{\color{bleu3}Méthode #1}}}}
{%
\end{bclogo}
\par}

%%%%%%%%%%%%% Mise en oeuvre
\newenvironment{MEO}[1][]{%

\begin{bclogo}[couleur=bleu1!15, arrondi =0.15, noborder=true, couleurBarre=bleu3, logo = \bccrayon ]{ 
\normalsize{{\color{bleu3}Mise en oeuvre #1}}}}
{%
\end{bclogo}
\par}


%%%%%%%%%%%%% Etymologie
\newenvironment{Ety}[1][]{%

\begin{bclogo}[couleur=olive!30, arrondi =0.15, noborder=true, couleurBarre=olive, logo = \bcplume ]{ 
\normalsize{{\color{olive}Étymologie#1}}}}
{%
\end{bclogo}
\par}


%%%%%%%%%%%%% Notation
\newenvironment{Nt}[1][]{%

\begin{bclogo}[couleur=red!30, arrondi =0.15, noborder=true, couleurBarre=red, logo = \bccrayon ]{ 
\normalsize{{\color{red}Notation#1}}}}
{%
\end{bclogo}
\par}
%%%%%%%%%%%%% Histoire
\newenvironment{His}[1][]{%

\begin{bclogo}[couleur=brown!30, arrondi =0.15, noborder=true, couleurBarre=brown, logo = \bcvaletcoeur ]{ 
\normalsize{{\color{brown}Histoire des mathématiques#1}}}}
{%
\end{bclogo}
\par}

%%%%%%%%%%%%% Attention
\newenvironment{Att}[1][]{%

\begin{bclogo}[couleur=red!30, arrondi =0.15, noborder=true, couleurBarre=red, logo = \bcattention ]{ 
\normalsize{Attention#1}}}
{%
\end{bclogo}
\par}

%%%%%%%%%%%%% A savoir
\newenvironment{AS}[1][]{%

\begin{bclogo}[couleur=red!30, arrondi =0.15, noborder=true, couleurBarre=red, logo = \bcattention ]{ 
\normalsize{A Savoir #1}}}
{%
\end{bclogo}
\par}

%%%%%%%%%%%%% Conséquence
\newenvironment{Cq}[1][]{%
\textbf{Conséquence#1}}
{%
\par}

%%%%%%%%%%%%% Vocabulaire
\newenvironment{Voc}[1][]{%
\setlength{\logowidth}{10pt}
\begin{footnotesize}
\begin{bclogo}[ noborder , couleur=white, logo =\bcbook]{#1}}
{%
\end{bclogo}
\end{footnotesize}
\par}


%%%%%%%%%%%%% Méthode en vidéo
\newcommand{\MtV}[2]{\vspace{0.4cm} \colorbox{bleu1!15}{\hspace{0.2 cm}\tikz\node[rounded corners=1pt,draw] {\color{red}$\blacktriangleright$}; \quad  \href{https://youtu.be/#1?rel=0}{\raisebox{0.8mm}{{\color{red}\textbf{Méthode en vidéo : #2}}}}}}


%%%%%%%%%%%%% A voir (AV) : Lien externe + vidéo non Youtube
\newcommand{\AV}[2]{\vspace{0.4cm} \colorbox{bleu1!15}{\hspace{0.2 cm}\tikz\node[rounded corners=1pt,draw] {\color{red}$\blacktriangleright$}; \quad  \href{https://youtu.be/#1?rel=0}{\raisebox{0.8mm}{{\color{red}\textbf{Méthode en vidéo : #2}}}}}}


%%%%%%%%%%%%% A voir (AV) : Lien externe + vidéo non Youtube
\newcommand{\LE}[2]{\vspace{0.4cm} \colorbox{bleu1!15}{\hspace{0.2 cm}\tikz\node[rounded corners=1pt,draw] {\color{red}$\blacktriangleright$}; \quad  \href{#1}{\raisebox{0.8mm}{{\color{red}\textbf{#2}}}}}}


%%%%%%%%%%%%% PESP  : Pour en Savoir Plus
\newcommand{\PESP}[1]{ \vspace{0.4cm}  \href{#1}{\raisebox{0.8mm}{{\color{brown}\textbf{>> Pour en savoir plus}}}}}




\newenvironment{ISN}[1][]{%
\vspace{0.4cm}
\begin{tcolorbox}[enhanced, lifted shadow={0mm}{0mm}{0mm}{0mm}%
{gray!60!white}, attach boxed title to top left={xshift=5mm, yshift*=-3mm}, coltitle=white, colback=gray!5!white, boxed title style={colback=gray!100}, colframe=gray!75!black,title=\textbf{Informatique et sciences du numérique#1}]}
{%
\end{tcolorbox}
\par}


%%%%%%%%%%%%% Video
\newenvironment{Vid}[1][]{%
\setlength{\logowidth}{12pt}
\begin{bclogo}[ noborder , couleur=white,barre=none, logo =\bcoeil]{#1}}
{%
\end{bclogo}
\par}


%%%%%%%%%%%%% Syntaxe
\newenvironment{Syn}[1][]{%
\begin{bclogo}[couleur=violet!15, arrondi =0.15, noborder=true, couleurBarre=violet!75, logo = \bcicosaedre ]{ 
\normalsize{{\color{violet!75}Syntaxe#1}}}}
{%
\end{bclogo}
\par}

%%%%%%%%%%%%% Logique
\newenvironment{Log}[1][]{%
\begin{bclogo}[couleur=bleu1!15, arrondi =0.15, noborder=true, couleurBarre=bleu3, logo = \bcloupe ]{ 
\normalsize{{\color{bleu3}Logique#1}}}}
{%
\end{bclogo}
\par}


%%%%%%%%%%%%% DTL 
\newenvironment{DTL}[1][]{%    \stepcounter{cpttheo} Théorème \thecpttheo . {#1}
\vspace{0.4cm}
\begin{tcolorbox}[enhanced, lifted shadow={0mm}{0mm}{0mm}{0mm}%
{bleu3!60!white}, attach boxed title to top left={xshift=5mm, yshift*=-3mm}, coltitle=white, colback=bleu1!5!white, boxed title style={colback=bleu3!100}, colframe=bleu3!75!black, title=  \textbf{Devoir en temps libre}{ #1}]}
{%
\end{tcolorbox}
\par}

%%%%%%%%%%%%% Auto évaluation
\newenvironment{autoeval}[1][]{%
\vspace{0.4cm}
\begin{tcolorbox}[enhanced, lifted shadow={0mm}{0mm}{0mm}{0mm}%
{black!60!white}, attach boxed title to top left={xshift=5mm, yshift*=-3mm}, coltitle=white, colback=sapgreen!5!white, boxed title style={colback=sapgreen!100}, colframe=sapgreen!75!black,title=\textbf{J'évalue mes compétences#1}]}
{%
\end{tcolorbox}
\par}


\newenvironment{autotest}[1][]{%
\vspace{0.4cm}
\begin{tcolorbox}[enhanced, lifted shadow={0mm}{0mm}{0mm}{0mm}%
{red!60!white}, attach boxed title to top left={xshift=5mm, yshift*=-3mm}, coltitle=white, colback=red!5!white, boxed title style={colback=red!100}, colframe=red!75!black,title=\textbf{Pour faire le point #1}]}
{%
\end{tcolorbox}
\par}


%%%%% Pour réinitialiser numéros des chapitres après une nouvelle partie
\makeatletter
    \@addtoreset{section}{part}
\makeatother

%%%%%%%%%%%%% Situation à programmer
\newenvironment{PC}[1][]{%
\begin{bclogo}[couleur=white!30, arrondi =0.15, noborder=true, couleurBarre=gray, logo = \bcquestion ]{ 
{\color{gray}\normalsize{Situation à programmer #1}}}}
{%
\end{bclogo}
\par}


%%%%%%%%%%%%% Ecrire le code
\newenvironment{Cod}[1][]{%
\begin{bclogo}[couleur=white!30, arrondi =0.15, noborder=true, couleurBarre=gray, logo = \bcrosevents ]{ 
\normalsize{Tester le code #1}}}
{%
\end{bclogo}
\par}

%%%%%%%%%%%%% Documentation
\newenvironment{Doc}[1][]{%
\medskip \begin{tcolorbox}[widget,colback=bleu2!15,colframe=bleu2!75!white,
adjusted title=  Documentation  . {#1} ]}
{%
\end{tcolorbox}\par}


%%%%%%%%%%%%% Problématiques
\newenvironment{Pbm}[1][]{%
\begin{bclogo}[couleur=bleu2!30, arrondi =0.15, noborder=true, couleurBarre=bleu3, logo = \bcinfo ]{ 
{\color{bleu3}\normalsize{Problématique#1}}}}
{%
\end{bclogo}
\par}

%%%%%%%%%%%%% Problématiques
\newenvironment{Sit}[1][]{%
\begin{bclogo}[couleur=bleu2!10, arrondi =0.15, noborder=true, couleurBarre=bleu3, logo = \bcquestion ]{ 
{\color{bleu3}\normalsize{Situation d'étude#1}}}}
{%
\end{bclogo}
\par}


%%%%%%%%%%%%% Script
\newenvironment{Script}[2][]{%     
\vspace{0.4cm}
\begin{tcolorbox}[enhanced, lifted shadow={0mm}{0mm}{0mm}{0mm}%
{bleu3!60!white}, attach boxed title to top left={xshift=5mm, yshift*=-3mm}, coltitle=white, colback=bleu1!5!white, boxed title style={colback=bleu3!100}, colframe=bleu3!75!black, title=  \textbf{#1}{ #2}]}
{%
\end{tcolorbox}
\par}




%%%%%%%% Entete
\newcommand{\entete}[3]{
%\noindent%
\textbf{#1} \hfill \textit{\textbf{#2}}
\vspace{-1em}
\begin{center}
\large{\textbf{\bsc{\fontfamily{cmr}\selectfont #3}}}
\end{center}
}

%%%%%%%%%%%%% Miniprojet 
\newenvironment{miniprojet}[1][]{%    \stepcounter{cpttheo} Théorème \thecpttheo . {#1}
\vspace{0.4cm}
\begin{tcolorbox}[enhanced, lifted shadow={0mm}{0mm}{0mm}{0mm}%
{bleu3!60!white}, attach boxed title to top left={xshift=5mm, yshift*=-3mm}, coltitle=white, colback=bleu1!5!white, boxed title style={colback=bleu3!100}, colframe=bleu3!75!black, title=  \textbf{Mini projet à rendre.}{ #1}]}
{%
\end{tcolorbox}
\par}


%%%%%%%% Activité
\newenvironment{Act}[1][]{%
\begin{bclogo}[couleur=white!30, arrondi =0.15, noborder=true, barre=snake, couleurBarre=green, logo = \bcpanchant ]{ 
\normalsize{Activité #1}}}
{%
\end{bclogo}
\par}

%%%%%%%%%%%%% Code sur console
\newtcolorbox{Console}{colback=black,colframe=black}

\newtcolorbox{Code}{colback=gray!20,colframe=gray!50}

%%%%% Spécial calculatrice %%%%%
\def\ecalc#1{{\textsf{\textbf{#1}}}}
\def\cvar{\fbox{$X,T,\theta$}}
\def\expos{\fbox{$\wedge$}}
\def\flhaut{\fbox{$\widehat{~~~~~}$}}
\def\flbas{\fbox{$~~~.~~$}}
\def\flgau{\fbox{$\langle$}}
\def\fldro{\fbox{$\rangle$}}

\usepackage{setspace}
%%%%%%%%%%%%%%%%%%%%%%%%%%%%%%%%%%%%%%%%%%%%%%%%%%%%%%%%%%%%
%%%%%%%%%%%%%%%%%%%%%%%%%%%%%%%%%%%%%%%%%%%%%%%%%%%%%%%%%%%%
%%%%%%%%%%%%%%%%%%%%%%%%%%%%%%%%%%%%%%%%%%%%%%%%%%%%%%%%%%%%
%%       Encadrement des chapitres pris à part.
%%%%%%%%%%%%%%%%%%%%%%%%%%%%%%%%%%%%%%%%%%%%%%%%%%%%%%%%%%%%
%%%%%%%%%%%%%%%%%%%%%%%%%%%%%%%%%%%%%%%%%%%%%%%%%%%%%%%%%%%%
%%%%%%%%%%%%%%%%%%%%%%%%%%%%%%%%%%%%%%%%%%%%%%%%%%%%%%%%%%%%

\makeatletter
\newlength{\temp@boite}
\newlength{\saveparindent}
\setlength{\saveparindent}{\parindent}
\def\breakboxparindent{\saveparindent}
\def\encadrement#1{%
  \def\bkvz@before@breakbox{\ifhmode\par\fi\vskip5pt\vskip\breakboxskip\relax}%
  \fboxrule=0.4pt
  \fboxsep=5pt
  \def\bkvz@set@linewidth{\advance\linewidth -2\fboxrule
                          \advance\linewidth -2\fboxsep}%
  \def\bkvz@left{\color{black}\vrule \@width\fboxrule\hskip\fboxsep\color{black}}%
  \def\bkvz@right{\color{black}\hskip\fboxsep\vrule \@width\fboxrule\color{black}}%
  \def\bkvz@top{\hbox to \hsize{%
      \color{black}%
      \setlength{\temp@boite}{\fboxrule+0.5ex}%
      \vrule\@width\fboxrule\@height \temp@boite %
      \rule[0.5ex]{2em}{\fboxrule}%
      {#1}%
      \setlength{\temp@boite}{\textwidth-2em-\widthof{#1}-2\fboxrule}%
      \rule[0.5ex]{\temp@boite}{\fboxrule}%
      \setlength{\temp@boite}{\fboxrule+0.5ex}%
      \vrule\@width\fboxrule\@height \temp@boite}}%
  \def\bkvz@bottom{\color{black}\hrule\@height\fboxrule}%
  \breakbox\vspace{0pt}}
\def\endencadrement{\vspace{3pt}\endbreakbox}
\makeatother




\makeatletter
\def\encadrementombre#1{%
  \fboxsep=5pt
  \fboxrule=0.4pt
  \def\bkvz@before@breakbox{%
    \ifhmode\par\fi\vskip5pt\vskip\breakboxskip\relax}%
  \def\bkvz@set@linewidth{%
    \advance\linewidth -2\fboxrule 
    \setlength{\fboxsep}{5pt}\advance\linewidth -2\fboxsep
  }%
  \def\bk@line{\hbox to \linewidth{%
      \ifbkcount\smash{\llap{\the\bk@lcnt\ }}\fi
      \setlength{\fboxsep}{0pt}\colorbox{white!5}{%
        \setlength{\fboxsep}{5pt}%
        {\color{black}\vrule width \fboxrule}\hskip\fboxsep
        \box\bk@bxa
        \hskip\fboxsep{\color{black}\vrule width\fboxrule}%
        }%
      }}%
  \def\bkvz@top{\hbox to \hsize{%
      \setlength{\temp@boite}{\fboxrule+0.5ex}%
      \color{white!5}\vrule\@width\textwidth\@height\temp@boite %
      \hspace{-\textwidth}%
      \color{black}%
      \vrule\@width\fboxrule\@height \temp@boite %
      \rule[0.5ex]{2em}{\fboxrule}%
      {\fboxsep=2pt\colorbox{white}{#1}}%
      \setlength{\temp@boite}{\textwidth-2em-\widthof{\fboxsep=2pt\colorbox{white}{#1}}-2\fboxrule}%
      \rule[0.5ex]{\temp@boite}{\fboxrule}%
      \setlength{\temp@boite}{\fboxrule+0.5ex}%
      \vrule\@width\fboxrule\@height \temp@boite %
}}%
  \def\bkvz@bottom{{\color{black}\hrule\@height\fboxrule}}%
  \color{black}\breakbox}%
\def\endencadrementombre{\endbreakbox}
\makeatother

\makeatletter
\def\encadrementombrevar#1{%
  \fboxsep=5pt
  \fboxrule=0.4pt
  \def\bkvz@before@breakbox{%
    \ifhmode\par\fi\vskip5pt\vskip\breakboxskip\relax}%
  \def\bkvz@set@linewidth{%
    \advance\linewidth -2\fboxrule 
    \setlength{\fboxsep}{5pt}\advance\linewidth -2\fboxsep
  }%
  \def\bk@line{\hbox to \linewidth{%
      \ifbkcount\smash{\llap{\the\bk@lcnt\ }}\fi
      \setlength{\fboxsep}{0pt}\colorbox{white!5}{%
        \setlength{\fboxsep}{5pt}%
        {\color{black}\vrule width \fboxrule}\hskip\fboxsep
        \box\bk@bxa
        \hskip\fboxsep{\color{black}\vrule width\fboxrule}%
        }%
      }}%
  \def\bkvz@top{\hbox to \hsize{%
      \setlength{\temp@boite}{\fboxrule+0.5ex}%
      \color{blue!5}\vrule\@width\textwidth\@height\temp@boite %
      \hspace{-\textwidth}%
      \color{black}%
      \vrule\@width\fboxrule\@height \temp@boite %
      \rule[0.5ex]{2em}{\fboxrule}%
      {\fboxsep=2pt\fcolorbox{black}{white!5}{#1}}%
      \setlength{\temp@boite}{\textwidth-2em-\widthof{\fboxsep=2pt\fcolorbox{black}{white!5}{#1}}-2\fboxrule}%
      \rule[0.5ex]{\temp@boite}{\fboxrule}%
      \setlength{\temp@boite}{\fboxrule+0.5ex}%
      \vrule\@width\fboxrule\@height \temp@boite %
}}%
  \def\bkvz@bottom{{\color{black}\hrule\@height\fboxrule}}%
  \color{black}\breakbox}%
\def\endencadrementombrevar{\endbreakbox}
\makeatother